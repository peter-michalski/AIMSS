\documentclass[12pt]{article}

\usepackage[round]{natbib}
\usepackage{booktabs}

\usepackage{hyperref}
\hypersetup{
    colorlinks=true,       % false: boxed links; true: colored links
    linkcolor=red,          % color of internal links (change box color with linkbordercolor)
    citecolor=blue,       % color of links to bibliography
    filecolor=magenta,   % color of file links
    urlcolor=cyan           % color of external links
}

%% Comments
\newif\ifcomments\commentstrue

\ifcomments
\newcommand{\authornote}[3]{\textcolor{#1}{[#3 ---#2]}}
\newcommand{\todo}[1]{\textcolor{red}{[TODO: #1]}}
\else
\newcommand{\authornote}[3]{}
\newcommand{\todo}[1]{}
\fi

\newcommand{\wss}[1]{\authornote{blue}{SS}{#1}} %Spencer Smith
\newcommand{\jc}[1]{\authornote{red}{JC}{#1}} %Jacques Carette
\newcommand{\oo}[1]{\authornote{magenta}{OO}{#1}} %Olu Owojaiye
\newcommand{\pmi}[1]{\authornote{green}{PM}{#1}} %Peter Michalski
\newcommand{\ad}[1]{\authornote{cyan}{AD}{#1}} %Ao Dong

%\oddsidemargin 0mm
%\evensidemargin 0mm
%\textwidth 160mm
%\textheight 200mm

\begin{document}

\title{Definining Quality} 
\author{Spencer Smith, Jacques Carette, Olu Owojaiye, Peter Michalski and Ao Dong}
\date{\today}
	
\maketitle

\begin{abstract}
  ...
\end{abstract}

\tableofcontents

\newpage

\section{Introduction} \label{SecIntroduction}

Purpose and scope of the document.

\section{Qualities} \label{SecQualities}

To assess the impact of MDE on SCS quality, we need a clear definition of what
we mean by quality.  The concept of
quality is decomposed into a set of separate qualities.  This set of qualities
can be applied to the software artifacts (documentation, test cases, etc) and to
the software development process itself.  Several of the qualities from the list
in Section~\ref{SecSoftwareQuality} cannot be measured directly, such as
maintainability.  Therefore, Section~\ref{SecDesirableQs} introduces measurable
documentation qualities that are believed to contribute to the overall software
qualities.  In some cases it may be necessary to use the qualities in
Section~\ref{SecDesirableQs} to indirectly measure qualities listed in
Section~\ref{SecSoftwareQuality}.

Our analysis is centred around a set of software qualities.  Quality is not
considered as a single measure, but a collection of different qualities, often
called ``ilities.''  These qualities highlight the desirable nonfunctional
properties for software artifacts, which include both documentation and
code. Some qualities, such as visibility and productivity, apply to the process
used for developing the software. The following list of qualities is based on
\cite{GhezziEtAl2003}. To the list from \cite{GhezziEtAl2003}, we have added
three qualities important for SC: installability, reproducibility and
sustainability.

\subsection*{Installability}

A measure of the ease of installation.


\item [\textbf{Correctness}] Software is correct if it matches its specification.

\item [\textbf{Verifiability}] involves ``solving the equations
  right''~\cite[p.~23]{Roache1998}; it benefits from rational documentation
  that systematically shows, with explicit traceability, how the governing
  equations are transformed into code.

\item [\textbf{Validatability}] means ``solving the right
  equations''~\cite[p.~23]{Roache1998}.  Validatability is improved by a
  rational process via clear documentation of the theory and assumptions, along
  with an explicit statement of the systematic steps required for experimental
  validation.

\item [\textbf{Reliability}] is a critical quality for scientific software,
  since the results of computations are meaningless, if they are not dependable.
  Reliability is closely tied to verifiability, since the key quality to verify
  is reliability, while the act of verification itself improves reliability.

\item [\textbf{Performance}] considerations can make certification challenging,
  since QA becomes more difficult for more complex code.  However, as
  Roache~\cite[p.~355]{Roache1998} points out, using simpler algorithms and
  reducing the number of options in general purpose code, is not always a
  practical option.

\item [\textbf{Usability}] can be a problem.  Different users, solving the same
  physical problem, using the same software, can come up with different answers,
  due to differences in parameter selection~\cite[p.~370]{Roache1998}.  To
  reduce misuse, a rational process must state expected user characteristics,
  modelling assumptions, definitions and the range of applicability of the code.

\item [\textbf{Maintainability}] is necessary in scientific software, since change,
  through iteration, experimentation and exploration, is inevitable.  Models of
  physical phenomena and numerical techniques necessarily evolve over
  time~\cite{CarverEtAl2007, SegalAndMorris2008}.  Proper documentation,
  designed with change in mind, can greatly assist with change management.%   QA
  % activities need to take the need for creativity into account, while not
  % smothering it~\cite[p.~352]{Roache1998}.

\item [\textbf{Reusability}] provides support for the quality of reliability,
  since reliability is improved by reusing trusted components~\cite{Dubois2005}.
  (Care must still be taken with reusing trusted components, since blind reuse
  in a new context can lead to errors, as dramatically shown in the Ariane 5
  disaster~\cite[p.~37--38]{OliveiraAndStewart2006}.)  The odds of reuse are
  improved when it is considered right from the start.

\item [\textbf{Understandability}] is necessary, since reviewers can only certify
  something they understand.  Scientific software developers have the
  view ``that the science a developer embeds in the code must be apparent to
  another scientist, even ten years later''~\cite{Kelly2013}.
  Understandability applies to the documentation and code, while usability
  refers to the executable software.  Documentation that follows a rational
  process is the easiest to follow.

\item [\textbf{Reproducibility}] is a required component of the scientific
  method~\cite{Davison2012}.  Although QA has, ``a bad name among
  creative scientists and engineers''~\cite[p.~352]{Roache1998}, the community
  need to recognize that participating in QA management also improves
  reproducibility.  Reproducibility, like QA, benefits from a consistent and
  repeatable computing environment, version control and separating code from
  configuration/parameters~\cite{Davison2012}.

\item [\textbf{Productivity}] \wss{This needs to be filled in.  Productivity is
    an important quality for this proposal, since part of what MDE promises is
    increased productivity.}

\item [\textbf{Sustainability}] \wss{This needs to be filled in.  A search needs
    to be done to find a good definition.  A starting point might be the recent
    paper that defines sustainability as a combination of other qualities.}

\end{description}

\subsection{Desirable Qualities of Documentation} \label{SecDesirableQs}

To achieve the qualities listed in Section~\ref{SecSoftwareQuality}, the
documentation should achieve the qualities listed in this section.  All but the
final quality listed (abstraction), are adapted from the IEEE recommended
practise for producing good software requirements~\cite{IEEE1998}.  Abstraction
means only revealing relevant details, which in a requirements document means
stating what is to be achieved, but remaining silent on how it is to be
achieved.  Abstraction is an important software development principle for
dealing with complexity~\cite[p.~40]{GhezziEtAl2003}.
\citet{SmithAndKoothoor2016} present further details on the qualities of
documentation for SCS.

\begin{description}

\item [\textbf{Completeness}] Documentation is said to be complete when all the
  requirements of the software are detailed. That is, each goal, functionality,
  attribute, design constraint, value, data, model, symbol, term (with its unit
  of measurement if applicable), abbreviation, acronym, assumption and
  performance requirement of the software is defined.  The software's response
  to all classes of inputs, both valid and invalid and for both desired and
  undesired events, also needs to be specified.

\item [\textbf{Consistency}] Documentation is said to be consistent when no subset
  of individual statements are in conflict with each other. That is, a
  specification of an item made at one place in the document should not
  contradict the specification of the same item at another location.

\item [\textbf{Modifiability}] The documentation should be developed in such a way
  that it is easily modifiable so that likely future changes do not destroy the
  structure of the document. Also it should be easy to reflect the change,
  wherever needed in the document to maintain consistency, traceability and
  completeness. For documentation to be modifiable, its format must be
  structured in a way that repetition is avoided and cross-referencing is
  employed.

\item [\textbf{Traceability}] Documentation should be traceable, as this
  facilitates maintenance and review. If a change is made to the design or code
  of the software, then all the documentation relating to those segments have to
  be modified.  This property is also important for recertification.

\item [\textbf{Unambiguity}] Documentation is said to be unambiguous only when
  every requirement's specification has a unique interpretation.  The
  documentation should be unambiguous to all audiences, including developers,
  users and reviewers.

\item [\textbf{Correctness}] There is no direct tool or method for measuring
  correctness. One way of building confidence in correctness is by reviewing to
  ensure that each requirement stated is one that the stakeholders and experts
  desire.  By maintaining traceability, consistency and unambiguity, we can
  reduce the occurrence of errors and make the goal of reviewing for correctness
  easier.

\item [\textbf{Verifiability}] Every requirement in the documentation must be the
  one fulfilled by the implemented software. Therefore all the requirements
  should be clear, unambiguous and testable, so that a person or a machine can
  verify whether the software product meets the requirements.

\item [\textbf{Abstract}] Documented requirements are said to be abstract if they
  state what the software must do and the properties it must possess, but do not
  speak about how these are to be achieved. For example, a requirement can
  specify that an Ordinary Differential Equation (ODE) must be solved, but it
  should not mention that Euler's method should be used to solve the ODE. How to
  accomplish the requirement is a design decision, which is documented during
  the design phase.

\end{description}

\newpage

\bibliographystyle {plainnat}
\bibliography {ResearchProposal}

\end{document}
